\documentclass{article}
\author{}

%Packages
\usepackage{amsmath}
\usepackage{mathabx}
\usepackage{dirtytalk}
\usepackage{textcomp}
\usepackage[margin=0.5in]{geometry}
\usepackage[none]{hyphenat}
\usepackage{biblatex}
%---
%Customize
\newcommand{\myparagraph}[1]{\paragraph{#1}\mbox{}\\}
%---

\title{Práctica: Expresiones Booleanas 1}

\hyphenpenalty 1
\exhyphenpenalty 1


\begin{document}
\maketitle

\begin{enumerate}
    \item Realizar las siguientes sustituciones textuales. Sea cuidadoso con la parentización. Remueva
los paréntesis innecesarios.

	\begin{enumerate}
		\setlength\itemsep{1em}
		\item $p[p := p \lor q]$
			\myparagraph{Solución:}
			\textlangle{} Aplicando sustitución textual, manteniendo parentesis de la variable sustituida \textrangle\par
			($p \lor q$)\par
			\textlangle{} Eliminando paréntesis redundantes \textrangle\par
			$p \lor q$\par

		\item $(p \lor q \Rightarrow q \lor p)[q := p \Rightarrow q]$
			\myparagraph{Solución:}
			\textlangle{} Sustitución textual de las ocurrencias de q \textrangle\par
			$(p \lor (p \Rightarrow q) \Rightarrow (p \Rightarrow q) \lor p)$\par
			\textlangle{} Eliminando paréntesis externos redundantes \textrangle\par
			$p \lor (p \Rightarrow q) \Rightarrow (p \Rightarrow q) \lor p$\par
			\textlangle{} Dado que la implicación es asociativa por la derecha el segundo \(p \Rightarrow q\) contiene paréntesis redundantes \textrangle\par
			$p \lor (p \Rightarrow q) \Rightarrow p \Rightarrow q \lor p$\par

		\item $((s \land q \equiv \neg r[r := s \lor q]) \Rightarrow p)[q := p \lor s]$
			\myparagraph{Solución:}
			\textlangle{} Sustitución de todas las ocurrencias de r \textrangle\par
			$((s \land q \equiv \neg (s \lor q)) \Rightarrow p)[q := p \lor s]$\par
			\textlangle{} Sustitución de todas las ocurrencias de q \textrangle\par
			$((s \land (p \lor s) \equiv \neg (s \lor (p \lor s))) \Rightarrow p)$\par
			\textlangle{} Eliminando paréntesis externos \textrangle\par
			$(s \land (p \lor s) \equiv \neg (s \lor (p \lor s))) \Rightarrow p$\par
			\textlangle{} Dado que \( \lor \) es asociativo: \(a \lor b \lor c \equiv ( a \lor b ) \lor c \equiv  a \lor (b  \lor c)\) \textrangle\par
			$(s \land (p \lor s) \equiv \neg (s \lor p \lor s)) \Rightarrow p$\par

		\item $(s[s := q \equiv w] \land true[p := w])[s := w \land x]$
		
		\item $(s \equiv p \land q)[s := p \lor r]$

	\end{enumerate}

    \item Realizar las siguientes sustituciones textuales simultáneas. Sea cuidadoso con la parentización.
Remueva los paréntesis innecesarios.
	\begin{enumerate}
		\item $p[p, q := p \lor q, q \land p]$
		
		\item $(p \lor q \Rightarrow q \lor p)[q, r := p \Rightarrow q, (p \land q) \lor (p \Rightarrow x \lor s)]$
		
		\item $((s \land q \equiv \neg r[r := s \lor q]) \Rightarrow p)[q, s := p \lor s, p \land q]$
		
		\item $(s[s := q \equiv w] \land true[p := w])[s, w := w \land x, s]$
		
		\item $(s \Rightarrow p \land q)[s, r := p \lor r, p \lor s]$
	\end{enumerate}

    \item Realizar las siguientes sustituciones textuales. Sea cuidadoso con la parentización. Remueva
los paréntesis innecesarios.

	\begin{enumerate}
		\item $p[p := p \lor q][q := q \land p]$
		
		\item $(p \lor q \Rightarrow q \lor p)[q := p \Rightarrow q][r := (p \land q) \lor (p \Rightarrow x \lor s)]$
		
		\item $((s \land q \equiv \neg r[r := s \lor q]) \Rightarrow p)[q := p \lor s][s := p \land q]$
		
		\item $(s[s := q \equiv w] \land true[p := w])[s := w \land x][w := s]$
		
		\item $(s \Rightarrow p \land q)[s := p \lor r][r := p \lor s][s := t][s := p \equiv v]$
	\end{enumerate}

	\item Para cada una de las expresiones E[z:=X] y \say{hints} X = Y, escriba la expresión resultante
E[z:=Y]. Puede haber varias respuestas, hallarlas todas.

	\begin{enumerate}
		\item E[z:=X]: $true$\par
		 X=Y: $true \equiv p \lor \neg p$

			\myparagraph{Solución:}
			\textlangle{} Tomando E: z \textrangle\par
			$E: z$\par
			\textlangle{} Con X \textrangle\par
			$X: true$\par
			\textlangle{} Con Y \textrangle\par
			$Y: p \lor \neg p$\par
			\textlangle{} Finalmente se tiene aplicando Leibniz \textrangle\par
			$E[z:=Y]: p \lor \neg p$\par
			\hspace{0.8cm}

		\item E[z:=X]: $false \lor p \equiv p$ \par
		 X=Y: $false \equiv \neg p \land p$
		 
			\myparagraph{Solución:}
			\textlangle{} La expresión E es de la forma \textrangle\par
			$E: z \lor p \equiv p$\par
			\textlangle{} Con X \textrangle\par
			$X: false$\par
			\textlangle{} Con Y \textrangle\par
			$Y: \neg p \land p$\par
			\textlangle{} Finalmente se tiene aplicando Leibniz \textrangle\par
			$E[z:=Y]: \neg p \land p \lor p \equiv p$\par
			\hspace{0.8cm}

		\item E[z:=X]: $p \lor (q \equiv p \lor q)$ \par
		 X=Y: $p \lor q \equiv q \lor p$
		 
			\myparagraph{Solución:}
			\textlangle{} La expresión E es de la forma \textrangle\par
			$E: p \lor (q \equiv z )$\par
			\textlangle{} Con X \textrangle\par
			$X: p \lor q$\par
			\textlangle{} Con Y \textrangle\par
			$Y: q \lor p$\par
			\textlangle{} Finalmente se tiene aplicando Leibniz \textrangle\par
			$E[z:=Y]: p \lor (q \equiv q \lor p )$\par
			\hspace{0.8cm}

		\item E[z:=X]: $(p \land (q \lor r)) \lor ((p \lor q) \land (p \lor r))$ \par
		 X=Y: $p \land (q \lor r) \equiv (p \land q) \lor (p \land r)$
		 
			\myparagraph{Solución:}
			\textlangle{} La expresión E es de la forma \textrangle\par
			$E: z \lor ((p \lor q) \land (p \lor r))$\par
			\textlangle{} Con X \textrangle\par
			$X: p \land (q \lor r)$\par
			\textlangle{} Con Y \textrangle\par
			$Y: (p \land q) \lor (p \land r)$\par
			\textlangle{} Finalmente se tiene aplicando Leibniz \textrangle\par
			$E[z:=Y]: (p \land q) \lor (p \land r) \lor ((p \lor q) \land (p \lor r))$\par
			\hspace{0.8cm}

		\item E[z:=X]: $p \Rightarrow (p \lor q \equiv p \land q)$ \par
		 X=Y: $p \equiv q \equiv p \lor q \equiv p \land q$
		 
			\myparagraph{Solución:}
			Esta pregunta tiene 3 respuestas que se muestran a continuación.
			\begin{enumerate}
				\item \textlangle{} La expresión E es de la forma \textrangle\par
				$E: z \Rightarrow (p \lor q \equiv p \land q)$\par
				\textlangle{} Con X \textrangle\par
				$X: p$\par
				\textlangle{} Con Y \textrangle\par
				$Y: q \equiv p \lor q \equiv p \land q$\par
				\textlangle{} Finalmente se tiene aplicando Leibniz \textrangle\par
				$E[z:=Y]: q \equiv p \lor q \equiv p \land q \Rightarrow (p \lor q \equiv p \land q)$\par
				\hspace{0.8cm}

				\item \textlangle{} La expresión E es de la forma \textrangle\par
				$E: p \Rightarrow (z)$\par
				\textlangle{} Con X \textrangle\par
				$X: p \lor q \equiv p \land q$\par
				\textlangle{} Con Y \textrangle\par
				$Y: q \equiv p$\par
				\textlangle{} Finalmente se tiene aplicando Leibniz \textrangle\par
				$E[z:=Y]: p \Rightarrow (q \equiv p)$\par
				\hspace{0.8cm}

				\item \textlangle{} La expresión E es de la forma \textrangle\par
				$E: p \Rightarrow (p \lor q \equiv z)$\par
				\textlangle{} Con X \textrangle\par
				$X: p \land q$\par
				\textlangle{} Con Y \textrangle\par
				$Y: p \equiv q \equiv p \lor q$\par
				\textlangle{} Finalmente se tiene aplicando Leibniz \textrangle\par
				$E[z:=Y]: p \Rightarrow (p \lor q \equiv p \equiv q \equiv p \lor q)$\par
				\hspace{0.8cm}
			\end{enumerate}

		\item E[z:=X]: $p \land (q \Rightarrow r) \equiv p \Rightarrow (q \Rightarrow r)$ \par
		 X=Y: $q \Rightarrow r \equiv p \lor q$

		\item E[z:=X]: $(p \lor q) \land r \equiv p \Rightarrow p \lor q$ \par
		 X=Y: $q \land r \equiv p \lor q$

		\item E[z:=X]: $p \lor (q \land r) \Rightarrow p \lor q$ \par
		 X=Y: $p \lor q \equiv \neg (\neg p \land q)$

		\item E[z:=X]: $(s \land \neg t) \lor (p \Rightarrow \neg t) \Rightarrow \neg t \equiv p \Rightarrow \neg t \lor q$ \par
		 X=Y: $p \Rightarrow \neg t \equiv \text{true}$

	\end{enumerate}

	\item Para cada una de las siguiente expresiones E[z:=X] y E[z:=Y], identifíque un \say{hints} X = Y
	que muestre que ellas son iguales e indique la E original.

	\begin{enumerate}
		\item  E[z:=X]: $(p \lor q \Rightarrow q \lor r) \land (p \lor q)$ \par
		 E[z:=Y]:$(p \Rightarrow (q \lor r)) \land p$ \par

		\item  E[z:=X]:$((\neg p \lor p) \lor (\neg p \lor q)) \Rightarrow (q \lor \neg p)$ \par
		 E[z:=X]:$((\neg p \lor p) \lor (p \Rightarrow q)) \Rightarrow (q \lor \neg p)$ \par

		\item  E[z:=X]:$(p \Rightarrow q) \Rightarrow (s \land t) \lor \neg r$\par
		 E[z:=X]:$(p \Rightarrow q) \Rightarrow \text{false} \lor \neg r$ \par

		\item  E[z:=X]:$r \equiv \neg s \equiv ( \text{true} \lor p \Rightarrow q) \land (r \equiv \neg s)$ \par
		 E[z:=X]:$r \equiv r \equiv \neg s \equiv ( \text{true} \lor p \Rightarrow q) \land (r \equiv r \equiv \neg s)$ \par
			
		 
		 \myparagraph{Solución:}
		 Esta pregunta tiene 2 respuestas que se muestran a continuación.
		 \begin{enumerate}
			 \item \textlangle{} La expresión E es de la forma \textrangle\par
			 $E: r \equiv z \equiv ( \text{true} \lor p \Rightarrow q) \land (r \equiv z)$\par
			 \textlangle{} Con X \textrangle\par
			 $X: \neg s$\par
			 \textlangle{} Con Y \textrangle\par
			 $Y: r \equiv \neg s$\par
			 \hspace{0.8cm}

			 \item \textlangle{} La expresión E es de la forma \textrangle\par
			 $E: z \equiv \neg s \equiv ( \text{true} \lor p \Rightarrow q) \land (z \neg s)$\par
			 \textlangle{} Con X \textrangle\par
			 $X: r$\par
			 \textlangle{} Con Y \textrangle\par
			 $Y: r \equiv r$\par
		 \end{enumerate}

	\end{enumerate}

	\item Elimine los paréntesis innecesarios de la siguientes expresiones.

	\begin{enumerate}
		\item $((p \Rightarrow q) \equiv (p \land (q \lor r)) \Rightarrow (r \lor (s \land t))) \lor \neg s \Leftarrow u \equiv t$
		
		\item $(r \lor (s \lor (t \lor \neg \neg q) \Rightarrow s) \equiv u) \equiv (p \lor q) \Rightarrow (t \equiv \neg t)$
		
		\item $((\text{true} \equiv (\neg \text{false} \Rightarrow \text{false}) \land \neg \text{true}) \Leftarrow \text{false}) \lor (\text{true} \land \text{false})$
		
		\item $((p \Rightarrow (q \Leftarrow r)) \lor (s \equiv t)) \lor (t \Rightarrow (\neg t \Rightarrow q) \equiv r) \lor (p \land p)$
		
		\item $p \Rightarrow ((q \land ((r \equiv q) \equiv t)) \lor \neg q) \equiv s \equiv p \lor q \Rightarrow t$
		
	\end{enumerate}

	\item Indique las todas las subexpresiones de la siguientes expresiones.

	\begin{enumerate}
		\item $p \land q \lor r \equiv p \Rightarrow r \lor q \land \neg(q \Rightarrow r \equiv s) \equiv a \lor b \Leftarrow s \equiv c$
		
		
		\myparagraph{Solución:}
		\textlangle{} Se definen las subexpresiones siguiendo la inducción estructural para las expresiones booleanas bien formadas, empezando por el caso base (constantes y variables), siguiendo con las expresiones negadas y finalmente las expresiones con operadores binarios en orden de precedencia \textrangle\par
		%\begin{sloopypar}
			\{p, q, r, s, a, b, c, ${\neg(q \Rightarrow r \equiv s)}$, ${p \land q}$, ${p \land q \lor r}$, ${r \lor q}$, $r \lor q \land \neg(q \Rightarrow r \equiv s)$, $a \lor b$, $p \Rightarrow r \lor q \land \neg(q \Rightarrow r \equiv s)$, $q \Rightarrow r$, $a \lor b \Leftarrow s$, $p \land q \lor r \equiv p \Rightarrow r \lor q \land \neg(q \Rightarrow r \equiv s)$, $q \Rightarrow r \equiv s$,  $p \land q \lor r \equiv p \Rightarrow r \lor q \land \neg(q \Rightarrow r \equiv s) \equiv a \lor b \Leftarrow s$, $p \land q \lor r \equiv p \Rightarrow r \lor q \land \neg(q \Rightarrow r \equiv s) \equiv a \lor b \Leftarrow s \equiv c$\}\par
		%\end{sloopypar}
		\hspace{0.8cm}
		

		\item $\neg a \land \neg b \equiv \neg(a \land b \equiv a \equiv b) \equiv c \lor d \Rightarrow e \land f \Leftarrow a \land b$

			\myparagraph{Solución:}
			\textlangle{} Se definen las subexpresiones siguiendo la inducción estructural para las expresiones booleanas bien formadas, empezando por el caso base (constantes y variables), siguiendo con las expresiones negadas y finalmente las expresiones con operadores binarios en orden de precedencia \textrangle\par
			\{ a, b, c, e, f, $\neg a$, $\neg b$, $\neg(a \land b \equiv a \equiv b)$, $\neg a \land \neg b$, $a \land b$, $c \lor d$, $e \land f$, $c \lor d \Rightarrow e \land f$, $c \lor d \Rightarrow e \land f \Leftarrow a \land b$, $a \land b \equiv a$, $a \land b \equiv a \equiv b$, $\neg a \land \neg b \equiv \neg(a \land b \equiv a \equiv b)$, $\neg a \land \neg b \equiv \neg(a \land b \equiv a \equiv b) \equiv c \lor d \Rightarrow e \land f \Leftarrow a \land b$ \}
			\hspace{0.8cm}

		\item $a \Rightarrow b \Rightarrow c \Rightarrow d \lor e \equiv f \land g \equiv h \Leftarrow i \lor j \lor k \lor \neg l \equiv m$
		
			\myparagraph{Solución:}
			\textlangle{} Se definen las subexpresiones siguiendo la inducción estructural para las expresiones booleanas bien formadas, empezando por el caso base (constantes y variables), siguiendo con las expresiones negadas y finalmente las expresiones con operadores binarios en orden de precedencia. Para este ejercicio que tiene 3 implicaciones consecutivas hay que tomar en cuenta que la implicación es asociativa por la derecha por lo tanto se debe proceder de derecha a izquierda \textrangle\par
			\{ a, b, c, e, f, g, h, i, j, k, l, m, $\neg l$, $d \lor e$, $f \land g$, $i \lor j$, $i \lor j \lor k$, $i \lor j \lor k \lor \neg l$, $c \Rightarrow d \lor e $, $b \Rightarrow c \Rightarrow d \lor e $, $a \Rightarrow b \Rightarrow c \Rightarrow d \lor e $, $h \Leftarrow i \lor j \lor k \lor \neg l$, $a \Rightarrow b \Rightarrow c \Rightarrow d \lor e \equiv f \land g$, $a \Rightarrow b \Rightarrow c \Rightarrow d \lor e \equiv f \land g \equiv h \Leftarrow i \lor j \lor k \lor \neg l$, $a \Rightarrow b \Rightarrow c \Rightarrow d \lor e \equiv f \land g \equiv h \Leftarrow i \lor j \lor k \lor \neg l \equiv m$  \}
			\hspace{0.8cm}

		\item $p \land q \lor r \Rightarrow \neg p \land q \lor r \lor s \equiv t \lor u \equiv \neg v \lor w \Leftarrow x \land y \land z \lor \neg(a \equiv b)$
		
		\item $\neg(a \lor b \equiv c \land d \Rightarrow \neg(e \Leftarrow f \equiv g)) \land r \lor s \equiv t \land v \lor \neg(a \equiv \neg b \land c \Rightarrow a)$
		
	\end{enumerate}

	\item Clasifíque las siguientes expresiones entre válidas, satisfacibles, contingencias e insatisfacibles.
	Justifíque su respuesta a través de una tabla de verdad.

	\begin{enumerate}
		\item $p \lor \neg q \Rightarrow q \land p \land \neg(q \land p)$

		\item $\neg(\neg(p \Rightarrow q)) \Rightarrow (p \ncong q)$

		\item $p \lor q \Rightarrow p \lor (q \equiv r) \land (p \lor r)$

		\item $p \Rightarrow (p \Rightarrow (q \equiv \text{true}) \Rightarrow r) \Rightarrow (p \lor \neg q)$

		\item $(p \ncong q) \land \neg r \Leftarrow \neg p \equiv p \lor (q \land \neg r)$
	\end{enumerate}

	\item Clasifíque las siguientes expresiones entre válidas, satisfacibles, contingencias e insatisfacibles.
	Justifíque su respuesta a través de una tabla de verdad.

	\begin{enumerate}
		\item $p \lor \neg q \equiv p \lor q \equiv p$

		\item $p \lor q \equiv \neg p \equiv q \equiv p \land q$

		\item $p \land (q \lor p) \equiv \neg q$

		\item $p \Rightarrow (q \Rightarrow p) \equiv p \Rightarrow q \Rightarrow p$ 
	\end{enumerate}

	\item Clasifíque las expresiones duales de las siguientes expresiones Booleanas en válidas, satis-
	facibles, contingencias e insatisfacibles. Justi?que en base a la clasi?cación dada para cada
	expresión dual, qué se puede decir de la expresión original.

	\begin{enumerate}
		\item $q \Rightarrow \neg p \equiv p \equiv q \equiv p \lor q$

		\item $q \lor p \equiv p \lor \neg q \equiv \neg p$

		\item $p \land (p \lor q) \equiv \neg p$

		\item $(p \lor \neg p) \land q \equiv \neg$ 
	\end{enumerate}

	\item Considere las siguiente expresiones Booleanas:
	\begin{enumerate}
		\item $(p \cdot q) * (q \oplus p)$

		\item $p \cdot q \equiv p \oplus \neg q \equiv \neg p$
		
		\item $(p \cdot (q \otimes r)) \oplus (q \odot p) * r$
	\end{enumerate}

	Sustituya los símbolos $\land$, $\lor$, $\oplus$, $\neg$, $\leftrightarrow$ por conectores del lenguaje de las expresiones Booleanas de manera que en cada sustitución cada expresión cumpla una de las siguientes condiciones:
	
	\begin{itemize}
		\item La expresión es una tautología.
		\item La expresión tiene al menos una valuación que la satisface y una que no.
		\item La expresión no tiene valuación alguna que la satisfaga.
		\item El negado de la expresión dual es una tautología.
	\end{itemize}

	Haga la tabla de la verdad para cada expresión obtenida.

	\item Sustituya los símbolos $\land$, $\oplus$, $\odot$ y $\otimes$ por conectores del lenguaje de las expresiones Booleanas de manera que la expresión resultante sea una tautología. ¿Qué puede decir sobre la validez de la expresión dual? Justifique su respuesta.\par
	$p \land q \oplus \neg(q \otimes \neg p) \land \neg t \odot t$

	\item ¿Son las siguientes expresiones Booleanas tautologías, contingencias o expresiones no satisfecibles? Justifíque su respuesta.

	\begin{enumerate}
		\item $((¬p \land q) \land (r \lor s)) \lor (¬t \land x) \Rightarrow (q \land ¬p \nleftrightarrow ¬k)$
		
		\item $(p \land q) \lor (¬q \land p) \Leftarrow p \lor (¬t \land x) \lor (u \nleftrightarrow v)$
	\end{enumerate}

\end{enumerate}

\end{document}